%%%%%%%%%%%%%%%%%%%%%%%%%%%%%%%%%%%%%%
\documentclass{beamer}
%%%%%%%%%%%%%%%%%%%%%%%%%%%%%%%%%%%%%%%
%% Load Packages
%%%%%%%%%%%%%%%%%%%%%%%%%%%%%%%%%%%%%%%

%%% Basic document setup %%%
\usepackage[utf8]{inputenc} % Encoding and language
\usepackage[english]{babel}
\usepackage{microtype} % Better typography
\usepackage[mdpgd-garamond]{mathdesign}
\usepackage{parskip}

%%% Mathematics packages %%%
\usepackage{amsmath}
\usepackage{amsfonts}
\usepackage{amssymb}
\usepackage{breqn}

%%% Define color hyperlinks %%%
\usepackage{xcolor}  % Paquete para manejar colores
\definecolor{CadetBlue}{RGB}{95, 158, 160} 
\usepackage{hyperref}  % Paquete para enlaces
\hypersetup{
    colorlinks=true, 
    linkcolor=blue,
    filecolor=blue,      
    urlcolor=blue,
    pdftitle={Overleaf Example},
    pdfpagemode=FullScreen
    }

%%%%%%%%%%%%%%%%%%%%%%%%%%%%%%%%%%%%%%%
%%% Figures and graphics
%%%%%%%%%%%%%%%%%%%%%%%%%%%%%%%%%%%%%%%
\usepackage{graphicx} % Including pictures
\usepackage{wrapfig} % In-line images
\usepackage{float} % Precise image placement
\usepackage{caption} % Subfigures
\usepackage{subcaption} % Subfigures
\usepackage{newfloat} % Custom floating environments
\DeclareFloatingEnvironment[name=Map]{map}
\captionsetup[figure]{labelfont=bf, labelsep=newline , name=Figure}
\captionsetup[map]{labelfont=bf, labelsep=newline}
\captionsetup{justification=centering}

%%%%%%%%%%%%%%%%%%%%%%%%%%%%%%%%%%%%%%%
%%% Tablas
%%%%%%%%%%%%%%%%%%%%%%%%%%%%%%%%%%%%%%%
\usepackage{booktabs}
\usepackage{array}
\usepackage{longtable}
\usepackage{multicol}
\usepackage{threeparttable}
\usepackage{caption}
\captionsetup[table]{labelfont=bf, labelsep=newline , name=Table}
\newcolumntype{L}[1]{>{\raggedright\let\newline\\\arraybackslash\hspace{0pt}}m{#1}}
\newcolumntype{C}[1]{>{\centering\let\newline\\\arraybackslash\hspace{0pt}}m{#1}}
\newcolumntype{R}[1]{>{\raggedleft\let\newline\\\arraybackslash\hspace{0pt}}m{#1}}


\usetheme{Boadilla}

%%%%%%%%% Title and First Slide %%%%%%%%%
\title[CIENFI]{Transformación Digital en el Sector Cooperativo: Finanzas Abiertas, Big Data y Machine Learning}
\author[U. Icesi]{Eduard F. Martínez González, Ph.D.}
\institute[]{Centro de Investigación en Economía y Finanzas (CIENFI) \\ Universidad Icesi}
\date{Marzo 2025}

%%%%%%%%%%%%%%%%%%%%%%%%%%%%%%%%%%%%%%
%%%%%%%%%%%%%%%%%%%%%%%%%%%%%%%%%%%%%%
\begin{document}

%%%%%%%%% Slide: Title %%%%%%%%%
\begin{frame}
\titlepage
\end{frame}

%%%%%%%%% Slide: Map Jhon Snow %%%%%%%%%
\begin{frame}{\textbf{Mapa de Colera Jhon Snow, 1854}}
\begin{center}
\includegraphics[scale=0.16]{figures/map_jhon_snow.jpg}
\end{center}
\textit{Nota:} Tomado de ..
\end{frame}

%%%%%%%%% Slide: Cooperativas de Ahorro y Crédito en Colombia %%%%%%%%%
\begin{frame}{\textbf{Cooperativas de Ahorro y Crédito en Colombia}}
\begin{center}
%\includegraphics[scale=0.24]{figures/cacs_colombia}
\end{center}
\scriptsize Fuente: Supersolidaria
\end{frame}

%%%%%%%%% Slide: Cooperativas de Ahorro y Crédito en Colombia %%%%%%%%%
\begin{frame}{\textbf{Colocación de Cartera en las CACs}}
\begin{center}
%\includegraphics[scale=0.18]{figures/cartera_bruta}
\end{center}
\scriptsize Fuente: Supersolidaria
\end{frame}

%%%%%%%%% Slide: Cooperativas de Ahorro y Crédito en Colombia %%%%%%%%%
\begin{frame}{\textbf{Excedentes y Perdidas en las CACs}}
\begin{center}
%\includegraphics[scale=0.17]{figures/excedentes_perdidas}
\end{center}
\scriptsize Fuente: Supersolidaria
\end{frame}

%%%%%%%%%%%%%%%%%%%%%%%%%%%%%%%%%%%%%%
%%%%%%%%%%%%%%%%%%%%%%%%%%%%%%%%%%%%%%
%%%%%%%%%%%%%%%%%%%%%%%%%%%%%%%%%%%%%%
\section{Motivación}
\section{Ecosistema Espacial en R}
\begin{frame}[noframenumbering]
\tableofcontents[currentsection]
\end{frame}

%%%%%%%%% Slide: Introducción a las Tendencias Globales %%%%%%%%%
\begin{frame}{\textbf{Tendencias Globales en Finanzas Digitales:}}
\begin{itemize}
    \item Crecimiento del uso de smartphones en operaciones bancarias:
    \begin{itemize} 
    \item 69\% de la población mundial tiene un smartphone (2023).
    \item 70\%-80\% de los usuarios de smartphones usan banca o pagos digitales.
    \end{itemize}
    
    \item Pagos digitales y banca móvil en expansión global:
    \begin{itemize} 
    \item Tarjetas (49\%) y billeteras digitales (32\%) dominan.
    \item Diferencias regionales: en Europa el efectivo aún es 22\%, en África y Medio Oriente supera el 40\%.
    \end{itemize}

    \item Cambios post-pandemia en hábitos financieros:
    \begin{itemize} 
   	\item Aceleró la adopción de pagos digitales y banca online. 40\% de adultos en economías emergentes usaron pagos digitales por primera vez.
    \end{itemize}
    
    \item Inversión Digital y Criptomonedas
    \begin{itemize}
    \item 20\% de las operaciones minoristas se hacen desde el móvil.
    \item 6-7\% de la población global posee criptomonedas (560M de personas en 2024).
    \end{itemize}
\end{itemize}
\end{frame}


%%%%%%%%% Slide: Directorio de Participantes %%%%%%%%%
\begin{frame}
    \begin{center}
    \Huge Gracias!
    \end{center}
\end{frame}

\end{document}
